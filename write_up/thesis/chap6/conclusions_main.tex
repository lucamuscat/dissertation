\chapter{Conclusions \& Future Work}
\section{Conclusion}
Limited work explicitly focusing on surveying concurrent
queueing algorithms exists. Consequently, we implement the algorithms
in~\citep{michael1996simple,hoffman2007baskets,valois1994queues} and compare
their performance through identical benchmarks. This study addresses the following research
objectives:

\paragraph{O1.} \emph{Implement a benchmarking framework for concurrent
queueing algorithms capable of gathering measurements similar to prior works;}

Section~\ref{sec:design_and_implementation_methodology} offers a detailed
description of the benchmarking methodologies used in this study; A high-level
overview of the benchmarking framework's design and implementation is provided in
section~\ref{sec:design_and_implementation_design_decisions}.

\paragraph{O2.} \emph{Reasonably validate the benchmarking framework through
metrics and experiments;}

Section~\ref{sec:design_and_implementation_validation} describes the efforts
taken in validating a number of the benchmarking framework's components.
The absolute error of the artificial delay (i.e.~the absolute difference
between the actual average and the expected time) and its coefficient of
variance are recorded and are kept to a minimum.
Although every queue's implementation is not thoroughly tested (in terms of
code coverage, due to the
complexity of testing concurrent algorithms), the frequency at which specific code
paths are executed is measured; specific distributions of these counters act as
a tell-tale sign that each algorithm is either working or not working as
expected. Finally, the repeatability of the benchmarking framework is
quantified through the results' coefficient of variance, where the $99\%$
quantile of the coefficient of variance for this study is $2.874\%$

\paragraph{O3.} \emph{Implement a variety of concurrent queueing algorithms, with the aim
of replicating results from the original works;}

Four queues~\citep{valois1994queues,hoffman2007baskets,michael1996simple} (three non-blocking, and one blocking) are implemented in this
study. The non-blocking queues require tagged pointers (pointers combined with a
version counter) for ABA avoidance purposes; implementations of both 64 bit and
128 tagged pointers are presented in this study (making a total of seven
queues).

\paragraph{O4.} \emph{Critically compare each concurrent queueing algorithm's performance
under a variety of synthetic benchmarks.}

Chapter~\ref{chap:evaluation} solidifies the study by offering multi-faceted
comparisons and interpretations of each queue's performance. None of the queues
in this study were able to outperform every other queue under all
circumstances, hinting at the fact there is no such thing as a
one-size-fits-all queue. Blocking queues outperform non-blocking queues at low
levels of contention (i.e. a combination of low thread count and high delay),
however, non-blocking queues still remain competitive. At high levels of
contention, blocking queues are inferior to non-blocking ones. 

Under workloads of three threads, the \citeauthor{michael1996simple} queue boasts an
impressive speedup of a factor less than $\frac{1}{3}$~\citep{michael1996simple},
however, \citeauthor{hoffman2007baskets}'s reported speedup for the
\emph{MS-Queue}~\citep{hoffman2007baskets} and the present study's reported
speedup are more modest.

We observe trends similar
to~\citep{ladan2008optimistic,hoffman2007baskets,michael1996simple}, where
every queue experiences a significant degradation in performance under two
threads, and a small speedup at three.

\citeauthor{michael1996simple}'s evaluation~\citep{michael1996simple}
introduces significant biases in favour of their novel algorithm through the
choice of memory-reclamation schemes (effectively increasing the magnitude by
which the \emph{MS-Queue} circumstantially outperforms every other queue).

The \emph{Baskets Queue's} performance is highly dependent on the
utilization of its `baskets' mechanism, which is triggered more often under
alternating enqueues and dequeues. \citeauthor{hoffman2007baskets}~\citep{hoffman2007baskets} fail to
discover this shortcoming due to their choice of benchmarking methodologies.
Consequently, the present study is only able to replicate
\citeauthor{hoffman2007baskets}'s results under the \emph{50\% Enqueue
Benchmark}.

Between one and three threads, the \emph{MS-Queue} consistently outperforms the
\emph{Baskets Queue}, however, only under the \emph{50\% Enqueue Benchmark} does the
\emph{Baskets Queue} start to consistently outperform the \emph{MS-Queue}
at four threads or above.

Queues using 128-bit tagged pointers (through the \emph{Double-Width
Compare-and-Swap} instruction) are several magnitudes slower than queues
adopting 64-bit pointers.

\section{Related Work}
\citeauthor{pourmeidani2019performance}~\citep{pourmeidani2019performance} evaluates the performance of a blocking
and a non-blocking (array-based) queue on a GPU, and finds that similar to
concurrent queues on CPUs, non-blocking queues require sufficient parallelism
to outperform blocking queues.

\citeauthor{gilbert2020performance}~\citep{gilbert2020performance}
uses several open source concurrent queueing algorithms and measures
their performance with the aim of finding the queue best suited for DBMS page
evictions.

\section{Critique and Limitations}
Our evaluation is limited by the number of cores on the benchmarking
machine's CPU, although this does not affect the validity of the study, it
potentially reduces the amount of insights found.

The queues evaluated in this study do not include memory-reclamation schemes,
potentially leading to biases.

Although several validation techniques are adopted, none of them are exhaustive
enough to ensure freedom from race conditions. The evaluation does not
discuss the benchmarking framework's overhead, making the degree of its effects
on the benchmarked code benchmarked unknown.

\section{Future Work}
\begin{itemize}
\item Add more algorithms with different progress conditions, such as
obstruction-free or wait-free queue;
\item Adopt \citeauthor{curtsinger2013stabilizer}'s \emph{stabilizer}~\citep{curtsinger2013stabilizer}, which allows
for statistically sound performance evaluations by forcing benchmark executions
to sample the space of memory configurations;
\item Further extend the study by running the benchmarks on UMA machines with
better processors;
\item Make the benchmarks more organic by substituting different queues in real
world code (similar to \citeauthor{boyd2012non}'s analysis of non-scalable locks~\citep{boyd2012non});
\item Add a variety of memory-reclamation schemes to the implemented queues,
discussing their effects on each queue's performance and behaviour.
\end{itemize}
