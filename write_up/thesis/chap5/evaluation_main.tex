\chapter{Evaluation}
In this chapter, insights emanating from each queue's evaluation is compared
with findings from their seminal works; each queue's performance is compared, 
together with explanations as to why and what makes a queue superior to another.


\section{Biases and Threats to Validity}
\paragraph{Veracity of Micro-Benchmarking Workloads}
In \citep{gregg2014systems}, \citeauthor{gregg2014systems} claims that
\emph{micro-benchmarks} (type of benchmark adopted in this study) produces
artificial workloads, rendering results which are  solely obtained under
various assumptions. Results obtained in this study represent the concurrent
queues, whose operations are separated by fixed delays, inside a ``clean-room''
environment. Realistic workloads seldom follow such rigid patterns, making the
results' validity under real-world scenarios undetermined.

\paragraph{Impact of Consumer Grade CPUs on Performance}
For the scope of this study, cutting-edge hardware, similar to that used in
prior art, is inaccessible. Unfortunately, over-subscription of processors
is required to gather enough an acceptable amount of data, making it
impossible to reproduce trends solely attainable from low-degrees of over-subscription. At
best, the resources accessible to the study may only produce
highly-multiprocessed workloads.

\section{Summary}