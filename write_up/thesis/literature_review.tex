\chapter{Literature Review}
\paragraph{Valois's Queue}
Valois covered multiple lock-free queue implementations that utilized linked lists, noting that most of the algorithms could be implemented using the Compare and Swap atomic primitive. Valois noted that CAS based algorithms suffer from the ABA problem. \cite{valois1994implementing}

Valois proposed the safe read protocol, which was a solution to the ABA problem that did not require stronger versions of CAS (such as DCAS) primitive \cite{valois1994implementing, valois1995lock}. A new lock free queue that made use of arrays and CAS was also proposed. The novel lock-free queue outperformed most lock-based queues both sequentially and under contention. Valois measured the performance of each lock-free queue with and without protection against the ABA problem, noting that even with the overhead, lock-free queues are still competitive with their lock-based counterparts.

Most notably, Valois measured the performance of each queue affected by random delay. Valois did this by simulating a delay of a random amount of cycles following an exponential distribution with a mean of 1000 cycles and a 10\% chance of occurring. The results showed that the response time of the lock-free implementations were approximately 17\% of that of the spin lock protected queues, highlighting the superiority of lock-based algorithms. The massive difference in performance was attributed to the blocking nature of the spin lock, noting that a delayed spin lock not only affected the current operation, but also the other pending critical sections.

Valois also concluded that spin lock algorithms require some type of back off mechanism in order to decrease the degraded performance caused due to interconnect saturation.

Performance analysis was conducted on a parallel architecture simulator called Proteus. The simulator was a work around to the infeasability of obtaining the necessary hardware and resources needed to obtain readings. Further exasperating this issue, not all machines used architectures that supported CAS primitives; most machines did not have good instrumentation facilities either. Porting code over to other architectures would have proved to be very time consuming \cite{valois1994implementing, valois1995lock}. 

Fortunately, these are mostly issues of the past, as hardware is more accessible and instrumentation facilities have matured over the last few decades.

\paragraph{The Baskets Queue}
Hoffman et. al propose a novel linearizable lock-free queue that takes advantage of the fact that the order of overlapping linearizable operations is non-deterministic \cite{hoffman2007baskets}. Each overlapping operation will be grouped together in a ``basket'' (list of nodes). Objects in a basket can be dequeued without order. A thread places an object into a basket when a compare and swap fails due to another thread being in progress.

\paragraph{Lock-free queue using elimination}
Shavit et. al introduce a novel lock-free queue using a technique known as elimination \cite{moir2005using}. Simply put, elimination works by allowing opposing operations such as enqueues and dequeues to exchange values without synchronizing on a shared data structure \cite{shavit1997elimination, moir2005using}. The results obtained suggests that as levels of concurrency increase, so does the throughput of the proposed data structure.

\subsubsection{Wait-free}
\label{wait-free-queues}

\paragraph{Kogan and Petrank wait-free queue}
\label{kp-queue}

Kogan et. al propose a novel and efficient wait-free queue that is on based link-lists and thread helping (ie. threads helping other threads by completing their intended operation for them)\cite{kogan2011wait}. Similar to the filter and the bakery locks, the queue is required to read and write to n distinct locations, where n is the maximum number of concurrent threads\cite{herlihy2020art}.
The proposed wait-free queue assumes that a garbage collector is responsible for preventing the ABA problem. Since a wait-free garbage collector does not exist, Kogan et al propose that the hazard pointer technique could be used to solve the ABA problem.

Performance of each queue mentioned in the paper was collected using the following methodology:
\begin{enumerate}
  \item enqueue-dequeue pairs: Starting with an empty queue, each thread iteratively performs an enqueue followed by a dequeue.
  \item 50\% enqueues: The queue is initialized with 1000 elements, each thread does a 'coin toss' to decide
\end{enumerate}