\chapter{Literature Review}
\label{chap:lit_review}

\section{Queues}
\subsection{\citeauthor{valois1994queues}' Queue}

\citeauthor{valois1994queues} surveys several lock-free data structures and
techniques \textemdash~ together with the introduction of the \emph{safe read}
memory reclamation scheme and an MPMC lock-free
queue~\citep{valois1994queues,valois1995datastructures}.
Combining the memory-reclamation scheme with the queue leads to increased cache line
churn, as enqueues traverse the queue's linked list for a bounded number of
nodes, causing each node's reference counter to be modified several times. 

% \citeauthor{michael1996simple} introduce the \emph{Two-Lock Concurrent} queue and the
% \emph{MS-Queue}~\citep{michael1996simple} \textemdash~a linked and concurrent
% queueing algorithm that is widely regarded as one of the most
% ubiquitous\footnote{Both \emph{Java\texttrademark's Concurrency
% Package}~\citep{java2022queue}, and \emph{Boost's Lock-Free
% Library}~\citep{boost2022queue} adopt the MS-queue algorithm.} lock-free
% algorithms in the field. Similar to \citeauthor{valois1994queues}' queue, the
% MS-queue uses disjoint head and tail references, together with
% the head always pointing at a dummy node~\citep{michael1996simple}. Thread
% helping is used as back-off technique that reduces contention from
% Compare-and-Swap.

\subsection{Michael's and Scott's Queues}

\paragraph{MS-Queue}
Similar to \citeauthor{valois1994queues}' queue, the MS-queue's head and tail
are separated, with the head always points to a dummy node. Compare-and-Swap
retry loops, which are a common pattern in lock-free algorithms, tend to
cause starvation and reduced parallelism.
\citeauthor{michael1996simple} adopt a thread-helping technique, where a thread
that fails to commit a node to the linked list, may help other threads by doing
part of their work, acting as a secondary back-off, reducing contention. 
The authors omit any discussion on the MS-queue's memory reclamation scheme,
which is a vital detail as pointed out by \citeauthor{michael2004hazard}
in \citep{michael2004hazard}, may lead to race conditions. 

\paragraph{Two-Lock Queue}
Similar to the MS-queue, the head and tail are separated, allowing for
concurrency between enqueues and dequeues~\citep{michael1996simple}. The ABA
problem does not exist in this algorithm, as Compare-and-Swap is not used;
furthermore, complex memory reclamation schemes are not needed, as access to
nodes are mutually exclusive, ensuring that a node may never be freed when
referenced by another thread.

\subsection{Optimistic MS-Queue}
\citeauthor{ladan2008optimistic} improve upon the MS-queue by enabling enqueues
to take effect in a single Compare-and-Swap~\citep{ladan2008optimistic}.
Enqueues add nodes to the beginning of the list;
doubly-linked lists allow for deleting nodes in the linked-list through backwards
traversals;
pointers to previous elements are maintained using simple stores, and are fixed
upon entering an inconsistent state (hence the \emph{optimistic} replacement of
Compare-and-Swaps).

\subsection{\citeauthor{hoffman2007baskets}'s Baskets Queue}
\citeauthor{hoffman2007baskets} present a variation of the MS-queue~\citep{hoffman2007baskets}, which is formed
using baskets (groups) of overlapping linearizable operations, which are non-deterministically
ordered among one another.
Time spent backing off in Compare-and-Swap failing threads is
spent inserting nodes into a basket, increasing parallelism across enqueuers;
the baskets mechanism also doubles down as a secondary back-off, further reducing contention.

Tagged pointers are used for ABA avoidance; dequeued nodes are logically
deleted by setting a flag bit packed inside the node's ABA counter. 
As the number of logically deleted nodes grows greater than the number of
\emph{max hops} (an arbitrarily chosen constant), or a logically deleted node
points to the tail of the queue, the \emph{free-chain} method is used to swing
the queue's head to the next non-deleted node, and reclaims any logically
deleted node between the head and the tail.

Under high levels of concurrency, the authors boast of a 25\% performance
improvement when compared to the MS-queue.

\subsection{Memory Reclamation Schemes}
Although memory reclamation schemes are omitted in this study, a brief discussion
on the field is essential to understanding the potential biases and discrepancies caused
by the omissions.

\subsubsection{Safe Read}
\emph{Safe Read}~\citep{valois1994queues,valois1995datastructures} 
is a reference counting memory management scheme that protects multiply referenced
pointers from ill-timed reclamation. 
Unlike the pointer packing technique, safe read reliably avoids ABA problems through
single-word Compare-and-Swap operations. \citeauthor{michael1995correction} discover
and correct race conditions in the \emph{safe read} protocol, that are liable to
corrupting and making unbounded use of bounded memory~\citep{michael1995correction}.

\subsubsection{Hazard Pointers}
\citeauthor{michael2004hazard}~\citep{michael2004hazard} presents a wait-free
memory reclamation methodology known as \emph{hazard pointers}

\section{Related Work}
Wait-free queues offer benefits such as starvation freedom and predictable
operation latencies \textemdash~at the cost of performance, practicality, and complexity.
Based on the MS-queue, \citeauthor{kogan2011wait} introduce one of the first practical unbounded, MPMC
wait-free queues~\citep{kogan2011wait} (henceforth known as the KP-queue). The
same authors present a methodology to create fast wait-free queues~\citep{kogan2012methodology} by
making use of the \emph{fast-path-slow-path} methodology; a lock-free queue is
used as the \emph{fast-path}, using a wait-free queue as a \emph{slow-path} after a number of failures.