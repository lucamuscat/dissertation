\chapter{Introduction}
\section{Problem Definition}

The act of writing and testing multiprocessor programs is, in jest,
oft-referred to as an art~\citep{herlihy2020art}, due to the difficulty of
ensuring the code's correctness. 
Due to the many ways to unknowingly cause
performance penalties or data corruption, intimate knowledge of the CPU's
architecture and memory is required. 

Within the realm of Computing Science, queues are among the data structures that
are utilized the most frequently. Since the development of multiprocessor
programming, concurrent queues have become an essential component of a wide
variety of concurrent and distributed algorithms.

\section{Aims and Objectives}

This study employs a benchmarking framework as well as a number of concurrent
queuing algorithms. In addition to the validation of results, all measurements
taken are compared to one another and to their original works. As a result, our
research objectives are as follows:

\begin{description}
\item[O1.] Implement a benchmarking framework for concurrent queueing
algorithms capable of gathering measurements similar to prior works;
\item[O2.] Reasonably validate the benchmarking framework through
metrics and experiments;
\item[O3.] Implement a variety of concurrent queueing algorithms, with the aim
of replicating results from the original works.
\item[O4.] Critically compare each concurrent queueing algorithm's performance
under a variety of synthetic benchmarks.
\end{description}

\section{Document Layout}
Chapter~\ref{chap:background} offers a brief overview of multiprocessor
programming concepts required to understand the contributions of this study; a
brief survey of the field of concurrent queueing algorithms is provided.
Chapter~\ref{chap:design_and_implementation} describes the concurrent queueing
algorithms in question and gives a high level overview of the benchmarking
framework.
Chapter~\ref{chap:evaluation} evaluates the performance of the
concurrent queueing algorithms implemented in this study, together with a
commentary on when one should use specific algorithms. The final chapter closes
the study by: giving an overview of the results obtained, discussing related
work, and offering future improvements that could be applied to the study.